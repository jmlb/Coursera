\documentclass[a4paper,12pt]{report}
\usepackage{multicol}
\usepackage[toc,page]{appendix}
\usepackage{amsmath}
\usepackage{float}
\usepackage{graphicx}
\usepackage{subfig}
\usepackage{amssymb}
\usepackage{geometry}
\usepackage{setspace}
\usepackage{soul}
\usepackage{tcolorbox}
\usepackage{multirow}% http://ctan.org/pkg/multirow
 \geometry{
 a4paper,
 total={170mm,257mm},
 left=20mm,
 top=20mm,
 }
\usepackage{tikz}
\usepackage{pgfplots}
\usetikzlibrary{shapes, arrows.meta, decorations.pathreplacing, positioning, petri, fit, calc}
\tikzstyle{startstop} = [rectangle, rounded corners, minimum width=2cm, minimum height=0.5cm,text centered, text width=3cm, draw=black, fill=gray!30]
\tikzstyle{process} = [rectangle, minimum width=2cm, minimum height=0.5cm, text centered, text width=3cm, draw=black, fill=blue!30]
\tikzstyle{detail} = [rectangle, minimum width=7cm, minimum height=0.5cm, text justified, text width=6.5cm, draw=black, fill=white!30]
\tikzstyle{smalldetail} = [rectangle, minimum width=3.5cm, minimum height=0.5cm, text justified, text width=3cm, draw=white, fill=white!30]
\tikzstyle{decision} = [rectangle, minimum width=3cm, minimum height=1cm, text centered, draw=black, fill=green!30]

\usepackage[utf8]{inputenc}

% Default fixed font does not support bold face
\DeclareFixedFont{\ttb}{T1}{txtt}{bx}{n}{10} % for bold
\DeclareFixedFont{\ttm}{T1}{txtt}{m}{n}{10}  % for normal

% Custom colors
\usepackage{color}
\definecolor{deepblue}{rgb}{0,0,0.5}
\definecolor{deepred}{rgb}{0.6,0,0}
\definecolor{deepgreen}{rgb}{0,0.5,0}

\usepackage{listings}

% Python style for highlighting
\newcommand\pythonstyle{\lstset{
language=Python,
basicstyle=\ttm,
otherkeywords={self},             % Add keywords here
keywordstyle=\ttb\color{deepblue},
emph={MyClass,__init__},          % Custom highlighting
emphstyle=\ttb\color{deepred},    % Custom highlighting style
stringstyle=\color{deepgreen},
%frame=tb,                         % Any extra options here
showstringspaces=false            % 
}}
\newcommand{\code}[1]{\texttt{#1} }

% Python environment
\lstnewenvironment{python}[1][]
{
\pythonstyle
\lstset{#1}
}
{}

% Python for external files
\newcommand\pythonexternal[2][]{{
\pythonstyle
\lstinputlisting[#1]{#2}}}

% Python for inline
\newcommand\pythoninline[1]{{\pythonstyle\lstinline!#1!}}

\begin{document}
\tableofcontents

\title{Data Scientist Toolbox \\
John's Hopkins University \\
Jeff Leek, Roger Peng, Brian Caffo}
\maketitle
\part{Week 1: What do data scientists do?}
\begin{itemize}
\item define the question
\item identify the ideal data set
\item gather data (from database or web)
\item clean the data
\item Perform exploratory analysis (plots, clustering)
\item Statistical prediction/modeling
\item interpret/challenge results
\item synthesize/write up results
\item create reproducible code
\item distribute results to others (interactive graphs, etc...)
\end{itemize}

\part{Command Line Interface (CLI)}
Use \textbf{Git Bash} from \textbf{Git} for windows to open a terminal.
\section{Command Line Interface}
CLI commands follow this recipe: \code{\textcolor[rgb]{0,0,1}{command} \textcolor[rgb]{1,0,0}{-flags} arguments}:
\begin{table}[H]
\centering
\begin{tabular}{|r|l|}
  \hline
  \code{command} & the CLI command which does a specifc task \\
	\hline
  \code{flags} & flags \\
	\hline
  \code{arguments} & binary \\

	\hline
\end{tabular}
\end{table}

\textbf{List of CLI commands:}\\
\begin{table}[H]
\centering
\begin{tabular}{|l|l|}
  \hline
  \code{\textcolor[rgb]{0,0,1}{pwd}} & print working directory \\
	\hline
  \code{\textcolor[rgb]{0,0,1}{ls}} & list files and folders in current directory \\
	\hline
  \code{\textcolor[rgb]{0,0,1}{ls} \textcolor[rgb]{1,0,0}{-a}} & list hidden/unhidden files/folders \\
	\hline
  \code{\textcolor[rgb]{0,0,1}{ls} \textcolor[rgb]{1,0,0}{-al}} & list details for hidden/unhidden files/folders \\
	\hline
	\code{\textcolor[rgb]{0,0,1}{cd} folder} & go to \textit{folder}\\
	\hline
	\code{\textcolor[rgb]{0,0,1}{cd}..} & change directory \\
	\hline
	\code{\textcolor[rgb]{0,0,1}{mkdir} fldr} & create folder \textit{fldr} \\
	\hline
	\code{\textcolor[rgb]{0,0,1}{touch} test\_file} & Create an empty file \\
	\hline
	\code{\textcolor[rgb]{0,0,1}{cp} fname fldr}  & copy file \textit{fname} in folder \textit{fldr} \\
	\hline
	\code{\textcolor[rgb]{0,0,1}{cp} \textcolor[rgb]{1,0,0}{-r} fldr1 fldr2}  & copy folder \textit{fldr1} to \textit{fldr2}  \\
	& (\code{\-r} flag stands for recursive) \\
	\hline
	\code{\textcolor[rgb]{0,0,1}{rm} \textcolor[rgb]{1,0,0}{-r} More\_docs} & Delete entire directories and its content \\
	\hline
	\code{\textcolor[rgb]{0,0,1}{mv} fname fldr1} & move file \textit{fname} to \textit{fldr1} \\
	\hline
	\code{\textcolor[rgb]{0,0,1}{mv} fname renfname} & rename file \textit{fname} to \textit{renfname} \\
	\hline
	\code{\textcolor[rgb]{0,0,1}{echo} Hello World!} & print \\
	\hline
	\code{\textcolor[rgb]{0,0,1}{date}} & current date \\ 
  \hline
\end{tabular}
\end{table}





\part{Git(local) and GitHub(remote)}
\section{Configuration: Username and Email}
Open \textbf{Git Batch}:
\begin{tcolorbox}
\begin{python}
$ git config --global user.name "Your Name Here"
$ git config --global user.email "Your email@Here.com"
#Confirm config
$ git --list #Return: (email/username)
\end{python}
\end{tcolorbox}

\section{Create a GitHub Repository}
\subsection{Start a repository from scratch}
\begin{table}[H]
\centering
\begin{tabular}{|r|l|}
  \hline
  1) & go to \textbf{Github.com} and click on new \\
	\hline
	2) & create a repository name and a brief description \\
		\hline
	3) & select \textbf{Public} \\
		\hline
	4) & check the box next to: "\textit{Initialize this repository with a README}" \\
  \hline \hline
\end{tabular}
\end{table}

\subsection{Create a local copy}
\begin{table}[H]
\centering
\begin{tabular}{|r|l|}
  \hline
	\hline
  1) & open \textbf{GitBash} \\
	  \hline 
	2) & create a directory where to store copy of the repository \\
	  \hline 
	3) & navigate to the directory \\
	  \hline 
	4) & check the box next to: "Initialize this repository with a README" \\
	  \hline 
	5) & initialize a local Git repository in this directory \\
	   & \code{\$ \textcolor[rgb]{0,0,1}{git} init } \\
	  \hline 
	6) & Point your local repository at the remote repository: \\
	   & \code{\$ \textcolor[rgb]{0,0,1}{git} remote add origin \textcolor[rgb]{0,0.58,0}{https://github.com/username/name\_repo.git}} \\
  \hline 
\end{tabular}
\end{table}

\subsection{"Fork" another user's repository}
Make a copy of the repository of someone else's:
\begin{table}[H]
\centering
\begin{tabular}{|r|l|}
  \hline 
1) & Go to the repository and click on Fork \\
\hline
2) & Make a local copy by "cloning" \\
\hline
3) & \code{\$ git clone \textcolor[rgb]{0,0.58,0}{https://github.com/username/name\_repo.git}} \\
  \hline 
\end{tabular}
\end{table}

\subsection{add to the index}
Adding (new files) to local repository
\begin{table}[H]
\centering
\begin{tabular}{|l|l|}
  \hline 
\code{\textcolor[rgb]{0,0,1}{git} add.} & adds all new files to be tracked by Git \\
\hline
\code{\textcolor[rgb]{0,0,1}{git} add \textcolor[rgb]{1,0,0}{-u}} & update tracking for files that changed names or were deleted \\
\hline
\code{\textcolor[rgb]{0,0,1}{git} add \textcolor[rgb]{1,0,0}{A}} & does both \\
  \hline 
\end{tabular}
\end{table}

\subsection{Commiting}
Commit to be saved as an intermediate version.
\begin{table}[H]
\centering
\begin{tabular}{|l|l|}
  \hline 
\code { \textcolor[rgb]{0,0,1}{git} commit \textcolor[rgb]{1,0,0}{-m} "message" } & \textit{message} is a description of what you did \\
& This only update local repo \\
  \hline 
\end{tabular}
\end{table}
\subsection{Pushing}
Update on remote (Github)
\begin{tcolorbox}
\begin{python}
$ git push
\end{python}
\end{tcolorbox}

\subsection{Branches}
Sometime, you are working on a project with a version being used by many people. You may not want to edit that version, you can create a \textbf{branch}: \\
\begin{table}[H]
\centering
\begin{tabular}{|l|l|}
  \hline 
\code{ \textcolor[rgb]{0,0,1}{git} checkout \textcolor[rgb]{1,0,0}{-b} \textcolor[rgb]{0.24,0.7,0.44}{branchename} }  & Create a branch \\
\hline
\code{ \textcolor[rgb]{0,0,1}{git} branch }  & check what branch you are on \\
\hline
\code{ \textcolor[rgb]{0,0,1}{git} checkout \textcolor[rgb]{0.24,0.7,0.44}{master} }  & switch to master branch \\
\hline
\end{tabular}
\end{table}
\subsection{Pull requests}
If you "fork" someone's repo or have multiple branches you will both be working separately. If you want to merge in your changes into the other branch: 
\begin{table}[H]
\centering
\begin{tabular}{|l|l|}
  \hline 
1) & Go to your branch \\
\hline
2) & Pull request (the request will be send to all parties) \\
  \hline 
\end{tabular}
\end{table}
\part{Markdown: .md}

\begin{tcolorbox}
\begin{python}
## This is a 2ndary heading
### This is a 3rdary heading
\end{python}
\end{tcolorbox}

\part{R for statistical computing}
Program: R and R-studio
\begin{tcolorbox}
\begin{python}
?rnorm  %access help file for function 'rnorm'
help.search("rnorm") %search help file
\end{python}
\end{tcolorbox}

\section{R-packages}
\begin{tcolorbox}
\begin{python}
a <- available.packages()  % check available packages
head(rownames(a),3) % show the names of the 1st 3 packages
install.packages("arg")  % install new package of name "arg"
install.packages(c("pack1", "pack2", "pack3"))  % install multiple packages
%%%%%%%%%%%%%%%%%%%%%%%%%%%%%%%%%%%%%%%%%%%%%%%%%%%%%
%go to Install packages in R-studio and load R-packages
library(packagename}
search()  % show all functions in package
\end{python}
\end{tcolorbox}

\subsection{RTools}
\textbf{R-Tools} is a collection of tools to build R-packages in Windows.\\
\begin{table}[H]
\centering
\begin{tabular}{|l|l|}
  \hline 
	\hline
1) & Open R Studio \\
\hline
2) & Install the devtools R package \\
\hline
3) & \code{find.package("devtools")} to check if package is installed ('True') \\
\hline
4) & \code{install.packages("devtools")} \\
\hline
5) & \code{library(devtools)} to load the package \\
\hline
	\hline
\end{tabular}
\end{table}
\end{document}
